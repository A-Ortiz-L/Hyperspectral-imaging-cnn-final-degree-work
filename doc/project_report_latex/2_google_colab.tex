\mbox{}


\chapter{Entrenamiento del modelo mediante Google Colab}
\label{ch:chapter2}
Uno de los parámetros más importantes dentro dentro del entrenamiento de modelos de deep learning es la velocidad.
La velocidad de entrenamiento es de vital importancia ya que los modelos pueden requerir entradas de tamaño masivo, por lo que la capacidad de cómputo
va a ser clave para acelerar este proceso y poder enfocarnos plenamente a la mejora del rendimiento de nuestro modelo, sin tener como cuello de botella la espera
que nos pueda producir volver a reentrenar el mismo con distintos parámetros.
De este modo podemos reajustar constantemente nuestro modelo para encontrar el punto óptimo de manera ágil.
En este trabajo vamos a entrenar nuestro modelo haciendo uso del framework de código abierto de Tensorflow, el cual incluye una API de deep learning llamada Keras, que será la que utilicemos.

El tipo de operaciones que requiere nuestra aplicación en la parte del tratamiento de imágenes y los propios procedimientos que realizan las redes neuronales
para hacer sus cálculos son operaciones matriciales.
Nuestro objetivo será aprovechar al máximo el rendimiento que una GPU nos puede dar en este tipo de operaciones gracias
a su arquitectura de paralelización, ya que es la idónea para este tipo de trabajo.
La ventaja que nos da frente a la cpu es la capacidad de cómputo, gracias a su conectividad por PCI express y el ancho de banda que esta proporciona.

https://www.nvidia.com/es-la/drivers/what-is-gpu-computing/

\section{Modelo propuesto}\label{sec:modelo-propuesto}
Modelo propuesto
\section{Entorno Google Colab}\label{sec:entorno-google-colab}
La plataforma de Google Colab es un servicio gratuito de google, mediante el cual podemos ejecutar e instalar librerías del lenguaje de programación python.
La ventaja de hacer uso de este medio es podemos configurar nuestro entorno para hacer uso de de herramientas industriales tales como una GPU Tesla K80.
Las características principales de nuestro principal unidad de cómputo son las siguientes :
4992 núcleos de NVIDIA CUDA con diseño de dos GPU