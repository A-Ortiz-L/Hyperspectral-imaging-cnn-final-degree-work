\cleardoublepage
%\newpage
%\thispagestyle{empty}
\mbox{}

\chapter{Conclusiones y trabajo futuro}
\label{ch:chapte6}

\section{Conclusiones}\label{sec:conclusiones}
La globalización de internet y el aumento de la población que interactúa con todas las aplicaciones conectados a el tiene como consecuencia
la generación de una cantidad infinita de datos que contienen toda la información de los usuarios que las usan.
Desde webs, aplicaciones móviles hasta sistemas que interactúan de manera automática sin necesidad de que los usuarios las configuren o las usen.
Ámbitos que hace hace unos pocos años no salían del ámbito académico y de estudio son ahora una necesidad real de mundo.
Conforme crecen los datos, crecen las necesidades de análisis y explotación.Plataformas como Google Cloud, Amazon Web Services y similares hacen esto posible
debido a toda la potencia autogestionada de cómputo que brindan a los usuarios.
Del mismo modo, nacen nuevas tecnologías de análisis específicas para estos datos.
Tecnologías como Tensorflow u OpenVINO pertenecen ambas a empresas comerciales como Google o Intel, sin embargo la creciente comunidad que hace uso de estas
herramientas de tal tamaño, que estas aplicaciones se han convertido al formato de código abierto, como hizo Google con Tensorflow, o directamente son gratuitas desde su lanzamiento, como OpenVINO.
Es la propia comunidad la que reporta errores, propone actualizaciones o codifica directamente nuevas soluciones para incorporar.
La competitiviadad entre todas estas soluciones impulsa una mejora constante en todas ellas.



\section{Trabajo futuro}\label{sec:trabajo-futuro}
La plataforma se ha configurado de manera que su capacidad de procesamiento de peticiones sea escalable, esto es, porque el uso de la tecnología de contenedores docker
está preparada para su integración en la solución de cluster Kubernetes\footnote{https://kubernetes.io/es/docs/concepts/overview/what-is-kubernetes/}.
La visualización de resultados en este trabajo ha sido realizada mediante los trabajos SQL realizados en la base de datos, pero sería ideal poder contemplar los datos de una manera más intuitiva.
Docker nos permite portar fácilmente esta solución de contenedores a otros sistemas con distinto hardware, en consecuencia, puede funcionar de manera local sin tener una plataforma web o cloud que la soporte.
La clasificación podría producirse dentro del propio elemento que genera las imágenes, lo que eliminaría el tiempo de latencia de otros elementos adicionales.
En nuestro problema un vehículo aéreo podría portar el propio hardware de clasificación y sobrevolar el terreno dañado para clasificar las imágenes que va recogiendo.
Este sistema sería similar a opciones que ya se pueden encontrar en el mercado, como las AWS DeepLens{https://aws.amazon.com/es/deeplens/} de Amazon.
Los datos de este trabajo son imágenes RGB de una parte del planeta, por lo que tienen latitud y longitud exactas, la técnica de visionado Heatmap\footnote{https://en.wikipedia.org/wiki/Heat_map} es la idónea para ver en tiempo real las zonas del terreno más afectadas por el desastre natural y poder redirigir los equipos de emergencia de manera rápida.
La configuración de los servidores ingesta las llamadas de manera interna, por lo que no está abierta a peticiones públicas de manera directa.
Si se decidiera abrir al público la clasificación de imágenes habría que configurar en los servidores todos los certificados necesarios para funcionar con el protocolo seguro HTTPS y medidas de seguridad contra factores como la denegación de servicio o intento de conexión no deseados a los servidores por los puertos abiertos de la aplicación, en nuestro caso 8080 y 22, que usamos para conectarnos por SSH\footnote{https://es.wikipedia.org/wiki/Secure_Shell}.

