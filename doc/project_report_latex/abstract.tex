% +--------------------------------------------------------------------+
% | Copyright Page
% +--------------------------------------------------------------------+

%\cleardoublepage 
%\newpage
%\thispagestyle{empty}

\begin{center}
{\bf \Huge Abstract}

  \end{center}

The remote observation of the Earth has always been an object of interest for the human being. Over the years, the methods used for this purpose have evolved until, at present, the analysis of hyperspectral images constitutes a very active line of research, especially to monitor fires or prevent and monitoring natural disasters, chemical discharges or other types of environmental pollution.

Due to the way in which materials appear in the natural environment, it is very common to cohabit different materials in the same portion of space, however small it may be. This means that the vast majority of pixels analyzed are not always constituted by the presence of a single material, but are formed by different pure materials at the sub-pixel level.

Traditionally, spectral demixing techniques are used for their analysis, which require two complex stages: the first is based on the extraction of pure spectral signatures, also called endmembers, and the second consists of estimating the percentage of abundance of said endmembers at the level of sub-pixel. Both stages involve a high computational cost and this is a problem when you want to analyze hyperspectral images in real time.

The algorithms of classification and detection of targets have some operating principles very similar to the detection algorithms of endmembers and therefore, they are usually used for this purpose despite their high computational cost. FPGAs (Field Programmable Gate Array) offer sufficient performance to make this processing as well as flexibility, small size and low consumption. All this together with the fact that they can be hardened for use in the space make this a very viable option for the objective pursued.

In this End-of-Degree Project, the FPGA implementation of the Automatic Target Detection and Classification Algorithm - Gram Schmidt known as ATDCA-GS, which uses the concept of orthogonal projection of a subspace, is carried out, using the orthogonalization of Gram Schmidt to simplify complex operations.

To achieve this proyect, the algorithm has been implemented in the hardware description language VHDL (Very High Speed Integrated Circuit Hardware Description Language) and in addition, another previous implementation under the OpenCL parallel programming paradigm was analyzed and optimized. Subsequently, both implementations optimized in terms of accuracy and performance have been compared on reconfigurable hardware platforms such as FPGAs.

%\vspace{1cm}



\vspace{0.8cm}

% +--------------------------------------------------------------------+
% | On the line below, replace Fecha
% |
% +--------------------------------------------------------------------+

\begin{center}

{\bf \Large Keywords}

   \end{center}

Hyperspectral images, ATDCA-GS, target detection, Gram Schmidt, FPGA, VHDL, OpenCL

   \vspace{0.5cm}
  
% Interactive whiteboard, 3D sensor device, human interface device, 2D positioning, signal processing, gesture recognition, microcontroller.
   
%\newpage
%\thispagestyle{empty}
\mbox{}
