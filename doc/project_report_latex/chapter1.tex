\cleardoublepage
\chapter{Introducción}
\label{ch:chapter1}

\section{Motivación}

Una imagen hiperespectral es una imagen de gran resolución espectral que se obtiene a través de sensores capaces de obtener cientos o incluso miles de imágenes sobre el mismo área terrestre pero correspondientes a diferentes canales de longitud de onda. El conjunto de bandas espectrales no está limitado estrictamente al espectro visible sino que también abarca el infrarrojo y el ultravioleta.

En la actualidad, el uso de imágenes hiperespectrales está aumentando considerablemente debido al lanzamiento de nuevos satélites y el interés en la observación remota de la Tierra, que tiene utilidad en ámbitos tan diversos como defensa, agricultura de precisión, geología (detección de yacimientos minerales), valoración de impactos ambientales o incluso visión artificial.

Durante los últimos años se han producido numerosos avances con respecto a la tecnología de los sensores, lo cual ha revolucionado la obtención, la manipulación y el análisis de los datos recopilados. Esta evolución ha conseguido que se pase de tener unas decenas de bandas a tener cientos y la tendencia es que el número siga aumentando. Instituciones como National Aeronautics and Space Administration (NASA) o la European Space Agency (ESA) están continuamente obteniendo una gran cantidad de datos que necesitan ser procesados. Como consecuencia, han surgido nuevos desafíos en el procesamiento de los datos.

%Para la observación remota de la Tierra, se han elaborado sensores de última generación que son ubicados en satélites y en plataformas aéreas para capturar datos de alta dimensionalidad.

Si al aumento de la cantidad de información recopilada le sumamos que muchas aplicaciones actuales y futuras de observación remota requieren capacidades de procesamiento en tiempo real (en el mismo tiempo o menos que lo que tarda el satélite en capturar los datos) o cercano a este tiempo real, se hace imprescindible el uso de arquitecturas paralelas para el procesamiento eficiente \cite{HPC_aplaza} y rápido de este tipo de imágenes.

El problema principal en el procesamiento de las imágenes hiperespectrales reside en la mezcla espectral, es decir, la existencia de píxeles mezcla en los que cohabitan varios materiales distintos a nivel de subpixel. Estos píxeles conforman la mayor parte de las imágenes hiperespectrales y para su análisis es preciso el uso de algoritmos complejos de alto coste computacional, que hacen que la ejecución de los algoritmos de desmezclado sea lenta y necesite una aceleración o paralelización.

Para abordar este tipo de tareas ha sido ampliamente utilizada la computación paralela a través de procesadores multi-cores, GPUs (Graphics Processing Units) o hardware dedicado como las FPGAs (Field-Programmable Gate Arrays). De todas las alternativas estas últimas presentan una opción eficiente en cuanto a rendimiento ofreciendo tiempos reducidos, además de una menor utilización de recursos siendo de las pocas alternativas que se pueden adaptar en un sensor para poder realizar un procesamiento a bordo en misiones espaciales como Mars Pathfinder o Mars Surveyor \cite{biblio:TFG_Esquembri}.

Por un lado, VHDL o Verilog son las maneras nativas de programar este tipo de dispositivos, a bajo nivel y más óptimas. Por otro lado, existe una alternativa en OpenCL que permite una programación a alto nivel, más rápida y permitiendo su ejecución en una variedad de arquitecturas pero menos óptima a nivel de recursos hardware que en los dispositivos FPGAs.

%[CARLOS - INICIO]
%Las ideas que deben aparacer aquí serían las siguientes:
    
%\begin{itemize}
    %\item Explicar muy brevemente qué son las imágenes hiperespectrales.
    %\item Cada vez se utilizan más las imágenes hiperespectrales. (se corresponde bien con tu primer párrafo pero podría ampliarse un poco y sobre todo poner referencias).
    %\item los sensores han evolucionado mucho y hemos pasado de tener unas decenas de bandas a tener cientos de bandas y la tendencia es que siga aumentando el número de bandas. (se corresponde bien con tu primer párrafo pero podría ampliarse un poco y sobre todo poner referencias).
    %\item Como consecuencia de este enorme aumento en la cantidad de información recopilada han surgido nuevos desafíos en el procesamiento de estos datos.
    %\item Hablar de la necesidad de ciertas aplicaciones de realizar el procesamiento en tiempo real (en el mismo tiempo o menos que lo que tarda el satélite en capturar los datos) o cercano a este tiempo real.
    %\item hablar del problema de la mezcla espectral y de que la ejecución de los algoritmos de análisis de desmezclado es lenta y necesita aceleración/paralelización.
    %\item Hablar de las ventajas de las FPGAs sobre otras tecnologías para procesar este tipo de imágenes y del modelo tradicional de programación en VHDL frente al uso de OpenCL. [Sergio-ini](Aquí puedes comentar que VHDL o Verilog es la manera nativa de programar este tipo de dispositivos, a bajo nivel y más óptima. Por otro lado, tienes OpenCL que permite una programación a alto nivel, más rápida, permitiendo su ejecución en una variedad de arquitecturas pero menos óptima a nivel de recursos hardware usados en un dispositivo FPGA.) [Sergio-fin]. 
%\end{itemize}
%[CARLOS - FIN]

\section{Objetivos}

El objetivo general de este trabajo es la implementación paralela sobre FPGA del algoritmo Automatic Target Detection and Classification Algorithm \cite{ATDCA,298007} haciendo uso de la ortogonalización de Gram Schmidt y de los lenguajes de programación VHDL y OpenCL. Esto permitirá una comparativa muy interesante entre un lenguaje nativo para dicha plataforma (VHDL) y otro paradigma de programación paralela a alto nivel (OpenCL) que podrá ser portado a otras plataformas como procesadores multi-cores, GPUs u otros aceleradores.

La consecución del objetivo general anteriormente mencionado se lleva a cabo en la presente memoria abordando una serie de objetivos específicos, los cuales se enumeran a continuación:

\begin{itemize}
    \item Diseño de módulos individuales en VHDL que sirvan para realizar todas las operaciones que se necesitan para la implementación del algoritmo ATDCA-GS.
    \item Elaboración de una máquina de estados e implementacion del algoritmo usando los modulos individuales.
    \item Análisis y optimización de una implementación paralela previa en OpenCL del algoritmo.
    \item Obtención de resultados y realización de comparativas de rendimiento entre ambos lenguajes de programación.
\end{itemize}

%[CARLOS - INICIO]

%Hablar del objetivo general de realizar la implementación paralela en FPGA del algoritmo Automatic Target Detection and Classification Algorithm haciendo uso de la ortogonalización deGram Schmidt para el análisis de imágenes hiperespectrales.

%La consecución del objetivo general anteriormente mencionado se lleva a cabo en la presente memoria abordando una serie de objetivos específicos, los cuales se enumeran a continuación:

%\begin{itemize}
    %\item
%\end{itemize}

%[CARLOS - FIN]

\section{Organización de esta memoria}

Teniendo presentes los anteriores objetivos concretos, se procede a describir la organización del resto de esta memoria, estructurada en una serie de capítulos cuyos contenidos se describen a continuación:

\begin{itemize}
    \item \textbf{Análisis hiperespectral}: se define el concepto de imagen hiperespectral y el modelo lineal de mezcla; se mencionan algunos sensores hiperespectrales (AVIRIS y EO-1 Hyperion) y algunas bibliotecas espectrales (USGS y ASTER); y por último, se presenta la necesidad de paralelización y las plataformas que se pueden utilizar para afrontar el problema de mejora de rendimiento.
    \item \textbf{Tecnología de las FPGAs}: se define de forma breve las tecnologías de las FPGAs.
    \item \textbf{Implementación}: se define el algoritmo ATDCA-GS en serie y se explica la paralelización y optimización que se ha llevado a cabo tanto en VHDL como en OpenCL.
    \item \textbf{Resultados}: se presentan los resultados obtenidos tras la implementación y ejecución del algoritmo en dispositivos FPGAs.
    \item \textbf{Conclusiones y trabajo futuro}: se presentan las principales conclusiones de los aspectos abordados en el trabajo a las que se han llegado y también algunas posibles líneas de trabajo futuro que se pueden desempeñar con relación al presente trabajo.
\end{itemize}

%[CARLOS - INICIO]

%Teniendo presentes los anteriores objetivos concretos, se procede a describir la organización del resto de esta memoria, estructurada en una serie de capítulos cuyos contenidos se describen a continuación:

%\begin{itemize}
    %\item
%\end{itemize}

%[CARLOS - FIN]