\cleardoublepage
\begin{center}

{\bf \Huge Resumen}

\end{center}

La observación remota de la Tierra ha sido siempre objeto de interés para el ser humano.
A lo largo de los años los métodos empleados con ese fin han ido evolucionando hasta que, en la actualidad, el análisis de imágenes multiespectrales constituye una línea de
investigación muy activa, en especial para realizar la monitorización y el seguimiento de incendios o prevenir y hacer un seguimiento de desastres naturales, vertidos químicos
u otros tipos de contaminación ambiental.

Las imágenes satelitales en un mundo donde el machine learning y el procesamiento de datos ha avanzado tanto nos abre la posibilidad de construir modelos capaces de reconocer zonas
en las que ha ocurrido un desastre natural y poder actuar en consecuencia.
Con la potencia de cálculo actual podemos conseguir que el procesamiento de los datos sea en tiempo real , por lo que se pueden tomar decisiones para poder minimizar daños, distribuir
equipos de emergencia y optimizar en general todos los recursos que tenemos.

Estos últimos años el campo de la ciencia de datos ha sido capaz de crear modelos precisos en sectores como banca, industria, tecnología.
Pero el círculo solo puede estar completo si podemos
conseguir colocar estos modelos en un entorno en el que se puedan consumir de manera productiva y ser útiles fuera del escenario de desarrollo e investigación.

Mantener toda la infraestructura hardware y software para ejecutar nuestra aplicación es costoso, de la misma manera que contratar a las personas con los conocimientos necesarios
para instalar y configurar todos estos componentes.
El mundo ha evolucionado de manera que ya es no necesario seguir este comportamiento y podemos optar por proveedores que facilitan todos estos componentes
así como el mantenimiento de los mismos, la nube.

En este trabajo de fin de grado se lleva a cabo la optimización en tiempos de inferencia de un modelo de machine learning usado para detectar desastres naturales con Openvino a la par que se realiza la puesta a
producción de la aplicación en un entorno cloud de Google, con el objetivo de que nuestro servicio soporte miles de peticiones por minuto.

\vspace{0.8cm}
\begin{center}


{\bf \Large Palabras clave}

\end{center}

Imágenes hiperespectrales, Openvino, Tensorflow, Docker, Google Cloud

\vspace{0.3cm}